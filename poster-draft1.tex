%%%%%%%%%%%%%%%%%%%%%%%%%%%%%%%%%%%%%%%%%%%%%%%%%%
%%%%%%%%%%%%%%%%%%%%%%%%%%%%%%%%%%%%%%%%%%%%%%%%%%
%%%%%%%%%%%%%%%%%%%%%%%%%%%%%%%%%%%%%%%%%%%%%%%%%%
% PREAMBLE:

\documentclass[12pt,a2paper,landscape]{article}

%%%%%%%%%%%%%%%%%%%%%%%%%%%%%%%%%%%%%%%%%%%%%%%%%%

\usepackage[noheadfoot,margin=10mm]{geometry}
% width=554mm,height=400mm
\usepackage[utf8]{inputenc}
\usepackage[english]{babel}
\usepackage{graphicx,multicol,color} 
%\usepackage{everyshi,eso-pic,calc,ifthen,wallpaper} 
\usepackage{tikz}
\usepackage{amsmath,amsfonts,amssymb,amsthm}
\usetikzlibrary{automata,arrows,positioning,calc}
\usepackage{mathtools}
\usepackage{caption}
\usepackage{anyfontsize}

%%%%%%%%%%%%%%%%%%%%%%%%%%%%%%%%%%%%%%%%%%%%%%%%%%

\pagestyle{empty}
\definecolor{cola}{rgb}{.99,.93,0}
\definecolor{colb}{rgb}{.8,0,.7}
\definecolor{colc}{rgb}{1,.9,.9}
\DeclarePairedDelimiter\abs{\lvert}{\rvert}

%% boxes to put stuff in ......
%%
%%   (1) transparent boxes --- the background shows through
%%			   #1 = width as fraction of what's available
%%                         #2 = contents (text, formula, picture)
%%                               ... tho pictures are always opaque
%% ---- box for headers --- width = fraction of textwidth 

\newcommand\BoX[2]{\begin{minipage}{#1\textwidth}#2\end{minipage}}
%% ---- box centred in a column --- width = fraction of columnwidth
\newcommand\BOX[2]{\begin{center}
   \begin{minipage}{#1\columnwidth}#2\end{minipage}\end{center}\vfill}
%%
%%   (2) opaque boxes with coloured background and outline frame
%%                         #1 = width as fraction of what's available
%%                         #2 = contents (text, formula, picture)
%%                         #3 = background colour
%%                         #4 = frame colour
\setlength\fboxrule{2pt} %% = width of frame lines round boxes
\setlength\fboxsep{5pt}  %% = spacing round box contents
%% ---- box for headers --- width = fraction of textwidth 
\newcommand\cBoX[4]{\fcolorbox{#4}{#3}{%
	\begin{minipage}{#1\textwidth}#2\end{minipage}}}
%% ---- box centred in a column --- width = fraction of columnwidth
\newcommand\cBOX[4]{\begin{center}\fcolorbox{#4}{#3}{%
 \begin{minipage}{#1\columnwidth}#2\end{minipage}}\end{center}\vfill}


%% for e.g. picture and text side-by-side in a column
%%   n.b. this command must be inside a \BOX or a \cBOX
\newcommand\sidebyside[2]{\BoX{.485}{#1}\BoX{.485}{#2}}

\def\bibsection{\section*{References}}        % Position reference section correctly

\renewcommand\familydefault{\sfdefault}

\newtheorem{definition}{Definition}[section]
\newtheorem{exmp}{Example}[section]
\newtheorem*{theorem*}{Theorem}
\DeclareMathOperator{\sech}{sech}		% Defining sech so it doesn't italicise it 
\DeclareMathOperator{\Int}{Int}		% Defining 'integer part of' so it doesn't italicise it 

\setlength\columnsep{10mm}  %%%% column separation

%%%%%%%%%%%%%%%%%%%%%%%%%%%%%%%%%%%%%%%%%%%%%%%%%%
%%%%%%%%%%%%%%%%%%%%%%%%%%%%%%%%%%%%%%%%%%%%%%%%%%
%%%%%%%%%%%%%%%%%%%%%%%%%%%%%%%%%%%%%%%%%%%%%%%%%%
% DOCUMENT

\begin{document}
%\tikz[remember picture,overlay] \node[inner sep=0pt] at (current page.center)
%{ \includegraphics[width=\paperwidth,height=\paperheight]{spiral3.pdf}};

\begin{tikzpicture}[remember picture, overlay]
  \node [anchor=north west, inner sep=1cm]  at (current page.north west)
     {\includegraphics[height=3cm]{university}};
\end{tikzpicture}
\hfill
\cBoX{0.75}{\centering
\vspace{1ex}
{\fontsize{1.8cm}{2cm}\selectfont{\textsc{Solitons in a Bose-Einstein Condensate \\}}}
\huge
\textsc{Lois P Flower\\ %L P Flower? 
Project Tutor: Dr. Rahul Sawant}
\vspace{1ex}
}{white}{white}
\hfill
%\cBoX{.3}{\color{black}
%\centering
%\huge
%\textsc{L P Flower\\
%Project Tutor: \\
%Dr. Rahul Sawant}
%}{white}{white}
\vspace{0.5in} 

%%%%%%%%%%%%%%%%%%%%%%%%%%%%%%%%%%%%%%%%%%%%%%%%%%
%%%%%%%%%%%%%%%%%%%%%%%%%%%%%%%%%%%%%%%%%%%%%%%%%%
%%%  now the body material in 3 columns of stacked boxes
%%      ... use \BOX and/or \cBOX here

\begin{multicols*}{3}
\setlength\fboxrule{1pt} 

%%%   to get more in, try say 4 columns with smaller-sized type
%%%  BUT -- for easy reading 
%%%               use no more than about 66 characters per line
%%%%   .... and choose a size of type that 
%%%%              looks OK expanded from A4 to A2
% \tiny
\small


%%%%%%%%%%%%%%%%%%%%%%%%%%%%%%%%%%%%%%%%%%%%%%%%%%

\fboxsep=5.5pt

\cBOX{0.97}{\section*{{\fontsize{28pt}{35pt}\selectfont Introduction}}
{
\fontsize{16pt}{20pt}\selectfont
The atoms in a Bose-Einstein condensate are all described by the same wavefunction, since they are in the same quantum state, which includes each atom's interaction with the other atoms in the condensate. Introducing an pseudopotential term $g \abs{\psi}^2$ which characterises the interactions between the particles into the dimensionless Schr\"{o}dinger equation yields the Gross-Pitaevskii equation \cite{}
\begin{equation}
i \frac{\partial \psi}{\partial t} = -\frac{\partial^2 \psi}{\partial x^2} + (V + g \abs{\psi}^2) \psi,
\end{equation}
where $\psi$ is the atomic wavefunction, $t$ and $x$ are the rescaled time and length variables, $V$ is the external potential (if present) and $g$ is the interaction parameter. 

It is usual to define $\zeta$ as a parameter to characterise width, with units of inverse length. The normalised wavefunction is then
\begin{equation} 
\psi(x) = \sqrt{\frac{\zeta}{2}} \sech{(\zeta x)} e^{i v x + \phi},
\end{equation}
where $v$ is the velocity of the soliton and $\phi$ is a phase factor. 
Substituting the $v=0$ case into the time-independent Schrodinger equation and setting E to be $\zeta^2$, we find $g=4\zeta$. 

\section*{{\fontsize{28pt}{35pt}\selectfont Propagating a Soliton}}

The Gross-Pitaevskii equation was solved numerically using the split-step Fourier method for rescaled time $t$ from 0 to either 20 or 40 [UNITS]. The box length used was initially 20 [UNITS], which was increased to 40 for repeated soliton collisions. [put in repeated collisions section?] A box length of 20 meant that a single soliton could have a velocity of up to 5 [UNITS] in the limit of zero width up to $t=20$, without boundary effects causing unrealistic distortion. 4000 space points and 4000 time points were used to minimise discretization effects. 

\begin{center}
\includegraphics[width=\columnwidth]{milestonepic.png}
\captionof{figure}{A graph showing the effect of inter-atom interactions $g \abs{\psi^2}$ on a soliton model.}
\end{center}
}
}{white}{white}

%%%%%%%%%%%%%%%%%%%%%%%%%%%%%%%%%%%%%%%%%%%%%%%%%%

\columnbreak
\cBOX{0.97}{\section*{{\fontsize{28pt}{35pt}\selectfont Soliton Collisions}}

\begin{center}
\includegraphics[width=\columnwidth]{extensionpic.png}
\captionof{figure}{Caption}
\end{center}


}{white}{white}

%%%%%%%%%%%%%%%%%%%%%%%%%%%%%%%%%%%%%%%%%%%%%%%%%%

\columnbreak
\cBOX{0.97}{\section*{{\fontsize{28pt}{35pt}\selectfont Repeated Collisions}}

{\fontsize{16pt}{20pt}\selectfont
Introducing a weak harmonic potential to the 1D Bose-Einstein condensate model confines the solitons axially within the box. After a collision the solitons climb a distance up the potential barrier dictated by their momentum (and hence velocity) and return to collide again. The axial potential preserves the relative phase and velocity of the solitons, allowing perfect periodic motion. However, the introduction of this potential places limits on the range of interaction parameters $g$ which lead to soliton-like behaviour.

\vspace{1ex}
\begin{center}
\includegraphics[width=\columnwidth]{difference-20.png}
\captionof{figure}{Two solitons with starting positions $-4$ and $4$ and velocities $20$ and $-20$, with relative phase either $0$ or $\pi$ radians. The interaction parameter $g=-12$ thus $\zeta=3$. The images here are plotted on a symmetric log scale with a linear cutoff of $\psi^2 = 0.05$ to improve visibility of the solitons. }
\end{center}

It was observed that for 

}
}{white}{white}

%%%%%%%%%%%%%%%%%%%%%%%%%%%%%%%%%%%%%%%%%%%%%%%%%%

\vspace{-2cm}
\cBOX{0.97}{\section*{Conclusion}
}{white}{white}

%%%%%%%%%%%%%%%%%%%%%%%%%%%%%%%%%%%%%%%%%%%%%%%%%%

\vspace{-1.9cm}
\cBOX{0.97}{

\begin{thebibliography}{}

\bibitem{KdV} D. J. Korteweg and G. de Vries (1895), "On the Change of Form...", \textit{Philosophical Magazine} \textbf{39}(240) pp. 422-443.

\bibitem{Segur} H. Segur (1973), "The Korteweg de Vries equation and water waves, part 1", \textit{Journal of Fluid Mechanics} \textbf{59} pp. 721-736.

\bibitem{Gardner} C. S. Gardner et al. (1967), "Method for Solving the Korteweg de Vries Equation", \textit{Physical Review Letters} \textbf{19}(19) pp. 1095-1097.

\end{thebibliography} 

}{white}{white}
\end{multicols*}
\end{document}

%%%%%%%%%%%%%%%%%%%%%%%%%%%%%%%%%%%%%%%%%%%%%%%%%%
%%%%%%%%%%%%%%%%%%%%%%%%%%%%%%%%%%%%%%%%%%%%%%%%%%
